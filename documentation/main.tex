\documentclass{article}
\usepackage[utf8]{inputenc}
\usepackage{verbatim}

\title{
  Västtrafik Reviewer \\
  \large Projektarbete av grupp 8 i kurs ENM156, Hållbar utveckling och etik inom datateknik }
  
\author{Alaa Eddin Alasiri, Felix Bråberg, \\Linus Johansson, Oscar Lilja, Rik Muijs, Philip Vedin}
\date{November 2020}


%======================/SAKER SOM SKA VARA MED I RAPPORTEN\======================
\begin{comment}
Generellt:
-Referera till kurslitteratur inom kursen
-Målgrupp = andra studenter i kursen
-Mellan 6-9 sidor
-Reflekterande text tydligen

Element som ska vara med:
-En kort beskrivning av er applikation.
-Reflektion kopplat till hållbarhet och etik.
-Hur skulle ni kunna göra bättre?
-Vilka är egna lärdomar?

För att rapporten ska var godkänd krävs att:
• Alla fyra punkter från listan ovan finns redovisade i rapporten
• Ni använder på ett korrekt sätt begrepp och teorier om hållbar utveckling och etik som presenteras i kurslitteraturen och föreläsningarna
• Ni skriver utifrån er själva och era egna erfarenheter
• Ni reflekterar över grunder och konsekvenser över ert förhållningssätt
• Ni väger olika argument mot varandra
• Rapporten är välskriven så att resonemangen går att förstå för nån som inte var med i gruppen 

\end{comment}
%======================\SAKER SOM SKA VARA MED I RAPPORTEN/======================

\begin{document}

\maketitle

\section*{Applikationen}
\textit{Västtrafik Reviewer} är en app för att betygsätta och skaderapportera hållplatser runtom i Sverige. Appen presenterar en interaktiv karta över Sverige där användaren kan trycka på ikoner som representerar diverse hållplatser i landet. Ett tryck på någon ikon presenterar ett fönster med information och bild på hållplatsen, se figur \ref{[LÄGG TILL BILD]}. Från den sidan kan användaren trycka på en knapp för att lämna en rapport om hållplatsen. På den sidan, figur \ref{[LÄGG TILL BILD]}, får användaren fylla i olika fält. Dessa fält är:
\begin{description}
\item[Kategori] som är vad för sorts rapport det är.
\item[Kommentar] som är kommentaren som användaren vill ge för platsen.
\item[Bild] som skickas om kategorin som är vald är \textit{Skadeanmälan}.
\item[Stjärnor] för att ge ett betyg för hållplatsen.
\end{description}

\subsection*{Målgrupp}
Den målgrupp som appen har som målgrupp är individer som reser med buss och tåg. 

\subsection*{Avgränsningar}
Det fanns många potentiella funktioner som hade kunnat implementerats och som även togs upp. Däremot, på grund av tidsbegränsningar, exkluderades dessa funktioner. Funktioner som exempelvis \textit{intsällningar} hade varit önskevärt för att erbjuda användaren större applikationsanpassning. Saker som olika färgteman och alternativa språk hade ökat tillgänglighetsaspekten av appen avsevärt. Dessutom fanns tekniska besvär som orsakade avgränsningar.
\\\\
Det var tänkt att alla hållplatser skulle hämtas från en API. Men nyckeln för att få tillgång till hållplatserna fungerade aldrig. Därför skapades istället en backend i programmet som innehåller ett antal hårdkodade hållplatser. 

\section*{Etik- och Hållbarhetsaspekter}


\section*{Förbättringsområden}
\section*{Reflektion}

\end{document}
