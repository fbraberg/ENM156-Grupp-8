\documentclass{article}
\usepackage[utf8]{inputenc}
\usepackage{verbatim}
\usepackage{graphicx}
\title{
  Västtrafik Reviewer \\
  \large Projektarbete av grupp 8 i kurs ENM156, Hållbar utveckling och etik inom datateknik }
  
\author{Alaa Eddin Alasiri, Felix Bråberg, \\Linus Johansson, Oscar Lilja, Rik Muijs, Philip Vedin}
\date{November 2020}


%======================/SAKER SOM SKA VARA MED I RAPPORTEN\======================
\begin{comment}
Generellt:
-Referera till kurslitteratur inom kursen
-Målgrupp = andra studenter i kursen
-Mellan 6-9 sidor
-Reflekterande text tydligen

Element som ska vara med:
-En kort beskrivning av er applikation.
-Reflektion kopplat till hållbarhet och etik.
-Hur skulle ni kunna göra bättre?
-Vilka är egna lärdomar?

För att rapporten ska var godkänd krävs att:
• Alla fyra punkter från listan ovan finns redovisade i rapporten
• Ni använder på ett korrekt sätt begrepp och teorier om hållbar utveckling och etik som presenteras i kurslitteraturen och föreläsningarna
• Ni skriver utifrån er själva och era egna erfarenheter
• Ni reflekterar över grunder och konsekvenser över ert förhållningssätt
• Ni väger olika argument mot varandra
• Rapporten är välskriven så att resonemangen går att förstå för nån som inte var med i gruppen 

\end{comment}
%======================\SAKER SOM SKA VARA MED I RAPPORTEN/======================

\begin{document}

\maketitle

\section*{Applikationen}
\textit{Västtrafik Reviewer} är en app för att betygsätta och skaderapportera hållplatser runtom i Sverige. Appen presenterar en interaktiv karta över Sverige där användaren kan trycka på ikoner som representerar diverse hållplatser i landet. Ett tryck på någon ikon presenterar ett fönster med information och bild på hållplatsen, se figur \ref{[LÄGG TILL BILD]}. Från den sidan kan användaren trycka på en knapp för att lämna en rapport om hållplatsen. På den sidan, figur \ref{[LÄGG TILL BILD]}, får användaren fylla i olika fält. Dessa fält är:
\begin{description}
\item[Kategori] som är vad för sorts rapport det är.
\item[Kommentar] som är kommentaren som användaren vill ge för platsen.
\item[Bild] som skickas om kategorin som är vald är \textit{Skadeanmälan}.
\item[Stjärnor] för att ge ett betyg för hållplatsen.
\end{description}

\subsection*{Målgrupp}
Den målgrupp som appen har som målgrupp är individer som reser med buss och tåg. 

\subsection*{Avgränsningar}
Det fanns många potentiella funktioner som hade kunnat implementerats och som även togs upp. Däremot, på grund av tidsbegränsningar, exkluderades dessa funktioner. Funktioner som exempelvis \textit{inställningar} hade varit önskevärt för att erbjuda användaren större applikationsanpassning. Saker som olika färgteman och alternativa språk hade ökat tillgänglighetsaspekten av appen avsevärt. Dessutom fanns tekniska besvär som orsakade avgränsningar.
\\\\
Det var tänkt att alla hållplatser skulle hämtas från en API. Men nyckeln för att få tillgång till hållplatserna fungerade aldrig. Därför skapades istället en backend i programmet som innehåller ett antal hårdkodade hållplatser. 

\section*{Etik- och Hållbarhetsaspekter}
\subsection*{Sociala aspekter}
Genom att se till att göra hållplatser säkrare främjas, på ett sätt, den sociala aspekten med kollektivtrafik. Att västtrafik tillhandahåller ett medel för deras resenärer att kommunicera önskemål och åsikter till företaget kan anses som en social aspekt av hållbarhet. Detta för att appen försöker främja resenärers förhållande i trafiken.

\subsection*{Ekonomiska aspekter}
Eftersom den utmaning som gruppen valde hade centrum i trygghet inom kollektivtrafiken har fokus inte varit på att adressera den ekonomiska aspekten. Dock påverkas de ekonomiska aspekten på ett indirekt sätt. 
\\\\
Genom att det mer enkelt finns tillgång till en plattform där individer kan lämna ett förslag eller skadeanmälan på en specifik hållplats kommer det finnas mer feedback på de olika hållplatser. Med denna information kan man sedan lätt förbättra dessa platser. De resurser som då används för att reparera en eventuell skada eller en modifikation av en hållplats kommer i sitt sätt leda till en utveckling inom kollektivtrafiken.
\subsection*{Ekologiska aspekter}

Applikationens syfte är främst att få folk att tycka till om Västtrafiks hållplatser, och på så sätt få en säkrare och trevligare miljö kring hållplatser. Således är den ekologiska aspekten en mer indirekt konsekvens. Vid stort användande av applikationen bör det kollektiva resandet öka, och på så sätt minska hur många som åker bil. Detta i sin tur leder till en minskning av koldioxidutsläpp.
\\\\
Förutom ökat kollektivt resande kan det även leda till ekologisk hållbarhet på andra sätt, ponera till exempel att det kommer in mycket kommentarer om att en hållplats alltid är full av skräp. Det kan mycket väl visa sig att det inte finns någon soptunna i närheten av hållplatsen, vilket kanske inte hade upptäckts annars och skräpet hade hamnat i naturen. 
\\\\
På detta sätt finns även en direkt möjlighet för Västtrafik att arbeta för att inte bara göra deras resor mer ekologiskt hållbara, utan de kan även göra individuella hållplatser och knytpunkter mer ekologiskt hållbara. Denna klimatpåverkan må vara nästintill försumbar i jämförelse med klimatpåverkan av att fler åker kollektivt, men samtidigt har Västtrafik 21 000 hållplatser - så totalt sett borde det göra en skillnad, även om den inte är lika omfattande. 

\subsection*{Etiska aspekter}

\section*{Förbättringsområden}
\subsubsection*{Tillägga alla busshållplatserna: }
Den aktuella versionen av applikationen innehåller hållplatserna till linje 18. Nästa steg ska vara att tillägga resten av busshållplatserna.
\\\\
\subsubsection*{Integrera med togo applikation: }
Ur användarens perspektiv anses Applikationen som komplettering tjänst till det huvud sakliga tjänst som är kollektivtrafik. Detta kan skapa svårighet att få stor antal användare. Lösning till detta kan vara att integrera vårt applikation med västtrafiks application"togo".
\\\\
*en bild från togo*
\\\\
Hållplatserna presenteras som noder i togo applikation.När man tryckar på en node, så visar appen hållplats namn. Det kan görs istället är att visa en feedback sida när man tryckar på en hållplats. Detta kan fungera på samma sätt som användt i vårt app.
\subsubsection*{poäng system: }
Att locka användaren att skriva en recension är på sig själv en utmaning, eftersom användaren inte får en direkt fördel genom att lämna in en utvärdering på en busshållplats. Genom att implementera ett poängsystem, kan vi uppmuntra användare att ladda ner och använda applikationen fortlöpande. Poängsystemet ger användaren fem poäng för en skadeanmäla som inkluderar en bild och kommentar, tre poäng för ett förslag och två poäng för ett omdöme. poäng som användaren samlar kan omvandlar till en reskassa. Detta kräver dock att ha en registrering system samt att ha en sida där man kan se sin aktuella samlade poänd och omvandla sina poäng till resekassa

\section*{Egna lärdomar}

\end{document}
